\chapter*{Introduction}
\markboth{Introduction}{Introduction}
\label{cha:introduction}
\addcontentsline{toc}{chapter}{Introduction}  

%---------------------------------------------------------------------------

% Aim - 
% Contents -

The aim of this work is to survey modern RGB-D image processing algorithms for model-based object recognition. Analysis of each method is focused on their usability in time and resource constrained robotic environment. Based on provided performance evaluation, selected methods are used to develop a complete, applicable in robotics, object recognition system.

The first chapter provides project background. A formal problem statement is included and followed by categorized solution proposals found in literature, outlining the scientific context of this work. The RGB-D imaging basics are subsequently introduced, together with software tools and test environment utilized throughout the project.

Second chapter introduces preprocessing techniques. Firstly, different data representation and conversion methods are discussed. Basic depth processing operations are introduced afterwards, including spatial transformations, neighbourhood selection and normal estimation. Different noise types encountered in RGB-D images and their corresponding filtering methods are further discussed.

Model fitting with sparse feature matching algorithms is presented in chapter three. Selected keypoint detectors and descriptors of both color and shape modalities are compared and an efficient matching technique is provided. Further, correspondence clustering and pose estimation methods are evaluated.
% Pose estimation may go into postprocessing

The following chapter is about 

%Model fitting algorithms are presented in chapter three. Performance of data alignment and correspondence matching methods is tested together with different descriptor approaches. Statistical methods of result verification are finally presented and used to test the complete object recognition system.

%Chapter four generalizes models from previously developed approach into categories. A neural network approach is presented as a data classifier. Convolutional neural network architecture is proposed and its implementation, training and performance tests are provided.

%The last chapter describes several application scenarios of the developed system. Here should be placed one sentence per scenario. Here should be placed one sentence per scenario. Here should be placed one sentence per scenario. Each tested against different noise conditions.

% 1st - RGB-D basics. Basics, Background, Problem background, Project backgrount, Project settlement, FIND EXAMPLE INTRODUCTIONS
%% Includes: color + depth intro, depth acquisition, work environment: software, hardware and datasets, and related work (pipelines).
% 2nd - Preprocessing
%% Includes: representation, neighbourhood, viewpoint transformations, noise filtering, segmentation, keypoint detection
% 3rd - Recognition
%% Includes: Alignment methods, matching, descriptors, hypothesis verification
% 4th - Classification
%% Includes: Convolutional neural network, training, tests
% 5th - Applications
%% Includes: Scenarios with simulation
