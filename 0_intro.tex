\chapter*{Introduction}
\markboth{Introduction}{Introduction}
\label{cha:introduction}
\addcontentsline{toc}{chapter}{Introduction}  

%---------------------------------------------------------------------------

% Aim - 
% Contents -

The aim of this work is to survey modern RGB-D image processing algorithms for model-based object recognition. Analysis of each method is focused on their usability in time and resource constrained robotic environment. Based on provided performance tests, selected methods will be used to develop a complete, applicable in robotics,object recognition system.

The first chapter provides project background. RGB-D imaging basics are introduced and followed by software tools and testing environment used throughout the project. Subsequently, scientific context of the work is settled.

Second chapter introduces preprocessing techniques. Firstly, different data representations and conversion methods are discussed. Afterwards, basic operations are introduced, including spatial transformations, neighbourhood calculation and normal estimation. Efficient noise filtering, data segmentation and keypoint extraction methods finalize this chapter.

Model fitting algorithms are presented in chapter three. Performance of data alignment and correspondence matching methods is tested together with different descriptor approaches. Statistical methods of result verification are finally presented and used to test the complete object recognition system.

Chapter four generalizes models from previously developed approach into categories. A neural network approach is presented as a data classifier. Convolutional neural network architecture is proposed and its implementation, training and performance tests are provided.

The last chapter describes several application scenarios of the developed system. Here should be placed one sentence per scenario. Here should be placed one sentence per scenario. Here should be placed one sentence per scenario. Each tested against different noise conditions.

% 1st - RGB-D basics. Basics, Background, Problem background, Project backgrount, Project settlement, FIND EXAMPLE INTRODUCTIONS
%% Includes: color + depth intro, depth acquisition, work environment: software, hardware and datasets, and related work (pipelines).
% 2nd - Preprocessing
%% Includes: representation, neighbourhood, viewpoint transformations, noise filtering, segmentation, keypoint detection
% 3rd - Recognition
%% Includes: Alignment methods, matching, descriptors, hypothesis verification
% 4th - Classification
%% Includes: Convolutional neural network, training, tests
% 5th - Applications
%% Includes: Scenarios with simulation
